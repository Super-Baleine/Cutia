\documentclass[a4paper,10pt]{article}
\usepackage[utf8]{inputenc}

\title{Cutia}
\author{Paul Feuvraux}

\begin{document}

\maketitle

THIS PROTOCOL IS STILL A DRAFT, EVERYTHING'S ISN'T FINISH YET, THERE ARE SEVERAL ELEMENTS I'D LIKE TO ADD.

\section{Introduction}
This paper aims to explain Cutia, a simple protocol comprehensive by everyone.

\section{Motivations}
I thought about this protocol because I was bored to see that all encrypted messaging protocol are always using SRP, HMAC, DH, and other things that aren't easy to understand for people who would like to design an encrypted messaging service. So I thought designing a protocol with AES and RSA only would be funny to do.

\section{Terms}
\begin{itemize}
    \item \textbf{Session}:
    Generic term used to talk about either a Conversation Session or a Group session.
    \item \textbf{Conversation Session ($CS$)}:
    A conversation session between two users.
    \item \textbf{Group Session ($GS$)}:
    A conversation session between more than users.
    \item \textbf{Key Agreement key ($KAK$)}:
    4096-bit asymmetric pair of RSA keys generated at the client-side.
    \item \textbf{Private Key ($PKA$)}:
    Used to decrypt the $SK$ of a $CS$. It is a part of the $KAK$.
    \item \textbf{Public Key ($PKB$)}:
    Used to encrypt the $SK$ of a $CS$ before exchanging it. It is a part of the $KAK$.
    \item \textbf{Session Key ($SK$)}:
    256-bit symmetric key used to encrypt and decrypt messages in a Session. It is generated at the client-side.
    \item \textbf{Passphrase ($P$)}:
    User's defined alphanumeric UTF-8 passphrase during registration on the client side.
    \item \textbf{Key Encryption Key ($KEK$)}:
    256-bit symmetric key used to encrypt the user's $CEK$.
    \item \textbf{Content Encryption Key ($CEK$)}:
    256-bit symmetric key randomly generated which is used to encrypt the user's $PKA$, and every $SK$ that he has to store.
    \item \textbf{Key derivation function}:
    Every derivation function is performed with PBKDF2. We denote this process as $KDF(x, s, i)$, where x is the passphrase, s the Salt, and i is the strengthen by factor and always equals 7000.
    \item \textbf{Symmetric Encryption function}:
    Every symmetric encryption is performed with AES under the Galois/Counter mode on 128-bit block cipher. We denote this process as $EncSym(k, x, i, t)$, where k is the symmetric key, x is the content to be encrypted, i is the initialization vector, and t is the authentication tag.
    \item \textbf{Symmetric Decryption function}:
    Every symmetric decryption is performed with AES under the Galois/Counter mode on 128-bit block cipher. We denote this process as $DecSym(k, x, i, t$, where k is the symmetric key, x is the encrypted content to be decrypted, is is the initialization vector, and t is the authentication tag.
    \item \textbf{Asymmetric Encryption function}:
    Every asymmetric encryption is performed with RSA. We denote this process as $EncAsym(k,x)$, where $k$ is a public key and $x$ is the content to be encrypted.
    \item \textbf{Asymmetric Decryption function}:
    Every asymmetric decryption is performed with RSA. We denote this process as $DecAsym(k, x)$, where $k$ is a private key and $x$ is the content to be encrypted.
\end{itemize}

\section{User}
\subsection{Registration}
To register, the user has to provide a Passphrase $P$. The user's $KAK$ is generated while this one is registering. The $PKA$ has to be encrypted before being sent to the server. $PKB$ is sent to the server and stored in plain text.
\subsubsection{$P$ derivation \& $CEK$ generation}
Once the user's Passphrase $P$ defined, a random Salt is generated on 128 bits. We proceed to a key derivation to obtain the user's $KEK$ such as $KEK=KDF(P, Salt, 7000)$. In the meantime, the $CEK$ is generated.
\subsubsection{Storage \& Encryption of $PKA$}
The user's $PKA$ has to be encrypted before being sent to the server. A random initialization vector (IV) and authentication tag (AT) are randomly generated on 128 bits. We proceed to the encryption of $PKA$ as $EncSym(CEK, PKA, IV, AT)$. Once encrypted, the encrypted $PKA$ is sent to the server with the IV, AT such as $PKA=(PKA||IV||AT)$.
\subsubsection{Storage \& Encryption of $CEK$}
A random IV and AT are generated on 128 bits. $CEK$ is encrypted under the $KEK$ such as $EncSym(KEK,CEK, IV, AT)$. Before sending $CEK$ to the server we encapsulate the cryptographic parameters IV, AT, the encrypted $CEK$ (eCEK) and the Salt such as $CEK=(eCEK||Salt||IV||AT)$. We're now able to send the encrypted $CEK$ to the server.

\subsection{Connection}
Once the user authenticated in the system of the application, the client gets user's $KAK$ and the $CEK$ which are stored encrypted in the server. The user types his Passphrase $P$.
\subsubsection{$P$ derivation}
From the encrypted $CEK$, we extract the Salt. We proceed to the derivation of the user's Passphrase $P$ such as $KEK=KDF(P, Salt, 7000)$.
\subsubsection{$CEK$ decryption}
We proceed to a symmetric decryption function to obtain the decrypted $CEK$ such as $CEK=DecSym(KEK, CEK, IV,AT)$.
\subsubsection{$PKA$ decryption}
From the $KAK$ we get the $PKA$ and decrypt it such as $DecSym(CEK, PKA, IV, AT)$.

\subsection{Change the user's $P$}
The user might need to modify his Passphrase $P$.
\subsubsection{Derivation of $P$}
The decrypted $CEK$ is needed. The user edits his Passphrase $P$. Once modified, the client proceeds to a derivation of $P$ to obtain the $KEK$. From the $CEK$ we get the Salt. Now the client is able to proceed to the derivation of $P$ such as $KEK=KDF(P, Salt, 7000)$.
\subsubsection{Encryption of $CEK$}
A random initialization vector and authentication tag are generated, both on 128 bits. The client proceeds to the $CEK$ encryption such as $EncSym(KEK, CEK, IV, AT)$. The client encapsulates the CEK and its cryptographic parameters such as $CEK=(CEK||Salt||IV||AT)$ and send the packet to the server.

\section{Paranoiac mode}
This mode is a $CS$ that doesn't support more than one device of every user and doesn't support any history. $SK$ is stored in the device, but isn't sent to the server.
\subsubsection{$SK$ exchange}
$SK$ is generated by one of the two participants (users) of the $CS$. The user's client that generated $SK$ gets the other user's $PKB$ and encrypt $SK$ under $PKB$ such as $EncAsym(PKB, SK)$. Once $SK$ encrypted, it is sent to the other user.
\subsubsection{Message Encryption}
For every message are generated an initialization vector (IV) and an authentication tag (AT), both are generated on 128 bits. Every message are encrypted by the Symmetric encryption function such as $EncSym(SK,message,Iv, AT)$. Once the message encrypted, we encapsulate the encrypted message $EM$ and the cryptographic parameters such as $m=(EM||IV||AT)$.

\subsubsection{Message Decryption}
The client gets the newly arrived message. From $m$ it extracts the encrypted message $EM$, the IV and the AT. The client proceeds to a symmetric decryption to get the plain text message $M$ such as $M=DecSym(SK, EM, IV, AT)$.

\section{Basic mode}
This mode is a $CS$ that supports keys history. This is useful to provide a multi-device messaging service.
\subsection{Structure of the history}
For every $SK$ an ID is generated, and every message encrypted under a certain $SK$ is composed of the ID of the $SK$. Then the history is composed of an or several encrypted $SK$ and their ID.
\subsection{$SK$ exchange}
$SK$ and its ID are generated by one of the two participants (users) of the $CS$. The user's client that generated $SK$ gets the other user's $PKB$ and encrypt $SK$ under $PKB$ such as $EncAsym(PKB, SK)$. Once $SK$ encrypted, it is sent to the other user such as $SK=(SK||ID)$.
\subsection{$SK$ encryption}
We generate a random initialization vector (IV) and a random authentication tag (AT), both on 128 bits. Every user's client encrypt the $SK$ under their $CEK$ such as $EncSym(CEK, SK, IV,AT)$. Once encrypted, they encapsulate the SK and its parameters such as $SK=(SK||ID||IV||AT)$ and they store this packet in the server.
\subsection{Entering a $CS$}
The client gets the encrypted $SK$ as a packet composed of the parameters of the $SK$. It extracts the IV, AT, and its ID. The client proceeds to a decryption of $SK$ such as $DecSym(CEK, SK, IV, AT)$ and decrypt the most recent encrypted messages which have been encrypted under the decrypted $SK$.
\subsection{Regenerating $SK$}
One of the two participants (users) generates a new $SK$ and its ID. The user who generated the new $SK$ gets the other user's $PKB$ and encrypt the $SK$ with the Asymmetric encryption function such as $EncAsym(PKB, SK)$ and encapsulate the encrypted $SK$ and its ID such as $SK=(SK||ID)$.

\subsection{Message encryption}
For every message are generated an initialization vector (IV) and authentication tag (AT), both on 128 bits. The plain text message $m$ is encrypted under $SK$ by the asymmetric encryption function such as $em=EncSym(SK, m, IV,AT)$, where $em$ is the encrypted message. Once encrypted, $em$ is encapsulated with its parameters such as $em=(em||IV||AT||ID)$.

\subsection{Message decryption}
The client extracts the cryptographic parameters from the message itself. The client decrypts the encrypted message $em$ under $SK$ such as $m=DecSym(SK, em, IV,AT)$.

\subsection{Browsing older messages}
The Id of $SK$ is not used to authenticate messages since we're using the Galois/Counter mode which is self authenticated. But the ID of the $SK$ is used to browse older messages which were encrypted under another $SK$.
\subsubsection{Getting the appropriate ID}
Every message is composed of an ID which corresponds with a $SK$ stored in the server. The client reads the ID of the message and gets the corresponding $SK$ to decrypt it.

\section{Group mode}
This mode is a $GS$.
\subsection{$SK$ exchange}
One of the participants (users) generates a new $SK$ and its ID. The user also gets the public keys of every participant and encrypts the message as many times as there are pubic keys ($PK$) such as $EncAsym(PK, SK)$ and encapsulate $SK$ and its ID as a packet $packet$ such as $packet=(SK||ID)$. Once this done, the user sends $packet$ to every participant.
\subsection{Joining a group}
The one's client who invited the user gets the $PKB$ of this user and encrypt the $SK$ of the group such as $EncAsym(PKB,SK)$. Once encrypted, $SK$ is put in a packet $packet$ such as $packet=(SK||ID)$.
\end{document}

